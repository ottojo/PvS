\documentclass{article}
\usepackage[utf8]{inputenc}
\usepackage{listings}


\usepackage[T1]{fontenc}
\usepackage{inconsolata}

\usepackage{color}

\definecolor{pblue}{rgb}{0.13,0.13,1}
\definecolor{pgreen}{rgb}{0,0.5,0}
\definecolor{pred}{rgb}{0.9,0,0}
\definecolor{pgrey}{rgb}{0.46,0.45,0.48}

\usepackage{listings}
\lstset{language=Java,
  showspaces=false,
  showtabs=false,
  breaklines=true,
  showstringspaces=false,
  breakatwhitespace=true,
  commentstyle=\color{pgreen},
  keywordstyle=\color{pblue},
  stringstyle=\color{pred},
  basicstyle=\ttfamily,
  moredelim=[il][\textcolor{pgrey}]{$$},
  moredelim=[is][\textcolor{pgrey}]{\%\%}{\%\%}
}

\title{PVS Blatt 12}
\author{Jonas Otto}
\date{Juli 2018}

\begin{document}

\maketitle

\setcounter{section}{2}
\section{Deadlocks}

\paragraph{a)}
Bei einer \lstinline{synchronized} Methode kann immer nur ein Thread auf das gesamte Objekt zugreifen, das Lock-Objekt ist dabei \lstinline{this}. Bei den \lstinline{synchronized} Blöcken hingegen ist ein Lock-Objekt explizit angegeben.

\paragraph{b)}
Es kommt an der Stelle \lstinline{synchronized (lock2)} zum Deadlock, nachdem beide Threads das erste \lstinline{lock1} haben.

\paragraph{c)}
Die Aufrufe von \lstinline{join()} auf allen Threads stellt sicher, dass das Programm erst beendet wird, wenn alle Threads jeweils beendet sind.

\paragraph{d)}

Damit die \lstinline{run()} Methode terminiert, muss der ausführende Thread sowohl \lstinline{lock1} und \lstinline{lock2} haben.

Der Deadlock tritt auf, wenn \lstinline{Thread t1} das Lock auf \lstinline{l1} hat, und \lstinline{Thread t2} das Lock auf \lstinline{l2} hat, also jeweils das lokale \lstinline{lock1}. Dabei hat der jeweils andere \lstinline{Thread} dann das lokale \lstinline{lock2}, was aber zum terminieren nötig ist.

\paragraph{e)}
Zum Verhindern des Deadlocks muss die Zeile \\
\lstinline{Thread t2 = new DeadLock(l2, l1);}\\
zu\\
\lstinline{Thread t2 = new DeadLock(l1, l2);}\\
geändert werden.

\end{document}
